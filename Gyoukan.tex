\documentclass[12pt, a4paper]{jsarticle}
    \usepackage{amsmath}
    \usepackage{amsthm}
    \usepackage[psamsfonts]{amssymb}
    \usepackage[dvipdfmx]{graphicx}
    \usepackage[dvipdfmx]{color}
    \usepackage{color}
    \usepackage{ascmac}
    \usepackage{amsfonts}
    \usepackage{mathrsfs}
    \usepackage{amssymb}
    \usepackage{graphicx}
    \usepackage{fancybox}
    \usepackage{enumerate}
    \usepackage{verbatim}
    \usepackage{subfigure}
    \usepackage{proof}
 

    %
    \theoremstyle{definition}
    %
    %%%%%%%%%%%%%%%%%%%%%%%%%%%%%%%%%%%%%%
    %ここにないパッケージを入れる人は,必ずここに記載すること.
    %
    %%%%%%%%%%%%%%%%%%%%%%%%%%%%%%%%%%%%%%
    %ここからはコード表です.
    %
   \newtheorem{axiom}{公理}[section]
    \newtheorem{defn}{定義}[section]
    \newtheorem{thm}{定理}[section]
    \newtheorem{prop}[thm]{命題}
    \newtheorem{lem}[thm]{補題}
    \newtheorem{cor}[thm]{系}
    \newtheorem{ex}{例}[section]
    \newtheorem{claim}{主張}[section]
    \newtheorem{property}{性質}[section]
    \newtheorem{attention}{注意}[section]
    \newtheorem{question}{問}[section]
    \newtheorem{prob}{問題}[section]
    \newtheorem{consideration}{考察}[section]
    \newtheorem{Alert}{警告}[section]
    \newtheorem{Rem}{注意}[section]
    %%%%%%%%%%%%%%%%%%%%%%%%%%%%%%%%%%%%%%
    %
    %定義や定理等に番号をつけたくない場合(例えば定理1.1等)は以下のコードを使ってください.
    %但し,例えば\Axiom*{}としてしまうと番号が付いてしまうので,必ず \begin{Axiom*} \end{Axiom*}の形で使ってください.
    \newtheorem*{axiom*}{公理}
    \newtheorem*{defn*}{定義}
    \newtheorem*{thm*}{定理}
    \newtheorem*{prop*}{命題}
    \newtheorem*{lem*}{補題}
    \newtheorem*{ex*}{例}
    \newtheorem*{cor*}{系}
    \newtheorem*{claim*}{主張}
    \newtheorem*{property*}{性質}
    \newtheorem*{attention*}{注意}
    \newtheorem*{question*}{問}
    \newtheorem*{prob*}{問題}
    \newtheorem*{consideration*}{考察}
    \newtheorem*{alert*}{警告}
    \newtheorem*{rem*}{注意}
    \renewcommand{\proofname}{\bfseries 証明}
    %
    %%%%%%%%%%%%%%%%%%%%%%%%%%%%%%%%%%%%%%
    %英語で定義や定理を書きたい場合こっちのコードを使うこと.
    \newtheorem{Axiom+}{Axiom}
    \newtheorem{Definition+}{Definition}
    \newtheorem{Theorem+}{Theorem}
    \newtheorem{Proposition+}{Proposition}
    \newtheorem{Lemma+}{Lemma}
    \newtheorem{Example+}{Example}
    \newtheorem{Corollary+}{Corollary}
    \newtheorem{Claim+}{Claim}
    \newtheorem{Property+}{Property}
    \newtheorem{Attention+}{Attention}
    \newtheorem{Question+}{Question}
    \newtheorem{Problem+}{Problem}
    \newtheorem{Consideration+}{Consideration}
    \newtheorem{Alert+}{Alert}
    %
    %
    %%%%%%%%%%%%%%%%%%%%%%%%%%%%%%%%%%%%%%
    %数
    \newcommand{\NN}{{\mathbb{N}}} %自然数全体,
    \newcommand{\ZZ}{{\mathbb{Z}}} %整数環
    \newcommand{\QQ}{{\mathbb{Q}}} %有理数体
    \newcommand{\RR}{{\mathbb{R}}} %実数体
    \newcommand{\CC}{{\mathbb{C}}} %複素数体
    
    \renewcommand{\thesection}{問\arabic{section}}
\renewcommand{\thesubsection}{(\arabic{subsection})}
\renewcommand{\thesubsubsection}{(\roman{subsubsection})}

    
    \title{ザ 行間 問題集}
    \author{@skbtkey}
    
    
    
\begin{document}
\maketitle
\abstract{このpdfは僕が勉強中に詰まった行間を問題の形に直して問題集として残しておくやつです.ご自由にお使いください.解答は気が向いたら書いておきます(\LaTeX 打つの面倒なので).日付が最終更新日になります}

\section{}
数列$\{ a_n \}$が$L (< \infty)$に収束するならば,
\[
\lim_{n \to \infty} \left( \frac{1}{n} \sum_{m=1}^{n}a_m \right) = L
\]
である.

\section{}
二つの位相空間$(X, \mathcal{O}_{X}),(Y,\mathcal{O}_{Y})$があり関数$f \colon X \rightarrow Y$が連続であること,すなわち,「任意の$Y$の開集合の$f$による逆像が$X$の開集合であること」と,「任意の$x \in X$に対して,任意の$f(x)$の近傍$V$をとると,ある$x$の近傍$U$が存在して,$U \subset V$」であることは同値.

\section{}
$G$を位相群,$H$は部分群とする.$a \in G$に対して,
\[
aT(xH) = axH , (xH \in G/H)
\]
と定めると,写像$aT \colon G/H \rightarrow G/H$は位相同型写像であることを示そう.ただし,$G/H$は$G$の$H$による左剰余類であり,商空間としての位相が入っているものとする.また,$G$から$G/H$への射影を$\phi$とする.
\subsection{}
$aT$はwell-definedか確かめよ.
\subsection{}
$aT$は単射であることを示せ(左剰余類の定義に沿って示して)
\subsection{}
$G/H$の開集合$U$に対して,$\phi^{-1}((aT)^{-1}(U)) = \phi^{-1}(U)$を示しなさい.これにより,連続写像であることが分かる.
\subsection{}
$aT$が同型写像であることを示しなさい.

\section{}
\[
e^{-\frac{\lambda}{q}(1-p^n)}e^{t(1-p^n)} \left(tp^n + \frac{\lambda}{q}(1-p^n) \right) ^y \frac{1}{y!}= \frac{1}{y!} e^{-\frac{\lambda}{q}(1-p^n)} \sum_{z=0}^{\min(x,y)} 
\]	

読んでいる本によって背景異なっているし,いちいち記号の定義書くのが面倒臭い(マルコフ連鎖ならちゃんと「マルコフ連鎖」と述べなきゃいけない.マルコフ連鎖の教科書を読んでいるなら文脈で判断できるが,このpdfではそういうは文脈が一切ないので問題ごとにいちいち記述しなければならない.).{\bf 無理では?}














\end{document}